\chapter{Introduction}
\section{Changing the Properties of Light}
Un faisceau laser peut être caractérisé par une intensité $I$, une puissance $P$, un rayon $w$, 
une longueur d'onde $\lambda$, une pulsation $\omega$, une polarisation et un vecteur de Poynting 
$\vec{S}$. Nous aimerions changer ces propriétés et la seule façon de faire est de la faire interagir 
avec un matériau.

\section{Introduction to Electrodynamics in Continuous Media}
\subsection{Maxwell's Equations}
Le champ électrique $\vec{E}(\vec r,t)$ et le champ magnétique d'induction $\vec B(\vec r,t)$ 
sont gouvernés par 
\begin{equation}
\begin{array}{ll}
\div \vec E &=\DS \frac{1}{\epsilon_0}\rho_a\\
\div \vec B &= 0\\
\vec\nabla\times\vec E&=\DS -\frac{\partial \vec{B}}{\partial t}\\
\vec\nabla\times\vec B&=\DS \mu_0\vec{J_a} + \epsilon_0\mu_0\frac{\partial \vec{E}}{\partial t}
\end{array}
\end{equation}
où $\rho_a$ (densité de charge (libre et polarisation) et $\vec{J}_a$ (courants : libres, polarisés 
et magnétiques) sont les termes sources. Dans le vide, les choses sont simples ($\vec E, \vec B$ 
continus et pas de termes sources) mais dans un matériau il faut prendre en compte toutes les 
particules pouvant contribuer au terme source. \\

On approxime souvent la densité de charge $\rho_a$ par une combinaison de charges séparées et de 
dipôles\footnote{Description utilisée lorsque deux charges égales mais de signe opposé sont 
considérées comme un ensemble.}
\begin{equation}
\vec p = q\vec d
\end{equation}
Par exemple, les atomes, molécules, \dots\ Pour les charge séparées on notera les électrons dans
un solide, les ions, \dots\ On note alors
\begin{equation}
\rho_a = \rho + \rho_P
\end{equation}\ \\

Le densité de courant totale $\vec{J_a}$ $[A/m^2]$ se note
\begin{equation}
\vec{J_a} = \vec J + \vec{J_P}+\vec{J_M}
\end{equation}
où $\vec{J}$ est le mouvement des charges libres, $\vec{J_P}$ la variation des dipôles et 
$\vec{J_M}$ la densité de courant magnétique. Notons la force de Lorentz agissant sur une 
charge $q$ à vitesse $\vec{v}$
\begin{equation}
f_{Lorentz} = q(\vec{E}+\vec{v}\times\vec{B})
\end{equation}
ainsi que la force agissant sur un dipôle électrique $\vec{P}$ (où $\vec{B}=\vec0$)
\begin{equation}
f_p = (p.\vec{\nabla})\vec{E}
\end{equation}
D'autres forces sont renseignées à la page 12.\\

Une approche (microscopique) pour étudier la propagation de la lumière dans la matière serait 
de résoudre ces équations couplées aux équations de mouvement de toutes les particules, ce qui 
n'est pas réaliste. Une autre approche est de voir les choses d'un point de vue macroscopique. 
Les équations de Maxwell sont valables aux dimensions macroscopique (nm) en considérant la moyenne
des champs sur un volume contenant un grand nombre d'atomes, ions, \dots\ Avec l'hypothèse de
continuité, on appelle cette approche \textit{electrodynamics of continuous media}.




\subsection{Polarisation Charge and Current}
Nous avons vu l'expression d'un dipôle microscopique ($\vec p = q\vec d$). Pour l'approche
macroscopique, il faut considérer la moyenne du moment dipolaire par unité de volume
\begin{equation}
\vec P(r) \equiv \lim\limits_{\Delta V\to \epsilon} \dfrac{\sum_i \vec p_i}{\Delta V}
\end{equation}
où $\vec PP$ est défini dans la limite où l'élément de volume $\Delta V$ autour de $\vec r$
est "petit" macroscopiquement, mais contient un grand nombre de dipôles. Le moment magnétique
d'un élément au point $\vec{r'}$ est donné par
\begin{equation}
P\Delta V'
\end{equation}
Celui-ci contribue au potentiel au point $\vec{r}$ d'une quantité (vecteurs à mettre)
\begin{equation}
\Delta U(r) = \frac{P(r')\Delta V'(r-r')}{4\pi\epsilon_0|r-r'|^3}
\end{equation}
Le potentiel au point $\vec{r}$ est alors donné par
\begin{equation}
U(\vec{r}) = \iiint_{V_0} \frac{P(r')(r-r')dV'}{4\pi\epsilon_0|r-r'|^3}
\end{equation}
Sachant que, par propriété$\nabla'(1/(|r-r'|)= (r-r')/(|r-r'|^3)$ et à l'aide du vecteur
différentiel $\nabla'(f\vec{P}) : f\nabla'.\vec{P}+\vec{P}.\nabla' f$, on peut ré-écrire le
potentiel
\begin{equation}
U(r) = \frac{1}{4\pi\epsilon_0}\iiint_{V_0} \dfrac{(-\nabla'.\vec{P})dV'}{|\vec{r}-\vec{r'}|} +
\frac{1}{4\pi\epsilon_0}\iint_{S_0}\dfrac{\vec{P}.dS}{|\vec{r}-\vec{r'}|}
\end{equation}
Par identification avec l'expression du potentiel du à une distribution de charge volumique ou 
surfacique et comme $[-\vec\nabla .\vec{P}] = [C/m^3]$ et $[\vec{P}] = [C/m^2]$, on peut écrire
\begin{equation}
\rho_P = -\vec\nabla .\vec{P}
\end{equation}
Sachant que le courant de polarisation est donné par\footnote{Cf. théorie microscopique des 
dipôles élémentaires}
\begin{equation}
\vec{J_p} = \dfrac{\partial\vec{P}}{\partial t}
\end{equation}
Compte-tenu de la précédente expression, nous obtenons
\begin{equation}
\dfrac{\partial\rho_p}{\partial t} = -\vec\nabla.\vec{J}_p
\end{equation}
La présence de la divergence se justifie par le fait que les charges sont séparées, le signe 
par convention. 

\subsection{Magnetisation Current}
Les courants de magnétisations sont des petites "boucles" magnétiques. Similairement à la 
sous-section précédente
\begin{equation}
M(\vec{r}) \equiv\lim\limits_{\Delta V\to \epsilon} \dfrac{\sum_i m_i}{\Delta V},\qquad
\qquad\vec{J_M} = \vec{\nabla}\times\vec{M}
\end{equation}

\subsection{Maxwell’s Equations Revisited}
En substituant dans les équations de Maxwell les résultats des deux précédentes sous-sections 
et en posant $\vec{D} = \epsilon_0\vec{E}+\vec{P}$ et $\vec H = \vec B/\mu_0-\vec{M}$, on 
trouve
\begin{equation}
\begin{array}{ll}
\div\vec{D} &=\rho\\
\div\vec{B} &= 0\\
\vec\nabla\times\vec{E} &=-\dfrac{\partial\vec{B}}{\partial t}\\
\vec\nabla\times\vec{H} &= \DS \vec{J}+\dfrac{\partial \vec{D}}{\partial t}
\end{array}
\end{equation}



\subsection{Constitutive Equations}
Pour avoir un système résolvable, il faut trouver une relation entre $\vec{P}, \vec{E}$ et 
$\vec{M},\vec{B}$. Ceci va être faisable avec une approché phénoménologique, en utilisant les 
équations constitutives. 

	\subsubsection{Champ de magnétisation}
	Dans ce cours, on considérera que $\vec{M} = \vec{0}$

	\subsection{Champ de polarisation}
	Beaucoup de matériaux contiennent des dipôles ou son polarisé à cause de $\vec{E}$. Ces 
	dipôles sont affectés lorsqu'une onde EM traverse le matériau : les particules positives
	bougent dans la direction du champ et les négatives dans le sens opposé. Il en résulte que
	$\vec{P} = f(\vec{E})$. On peut alors considérer le développement en série suivant 
	\begin{equation}
	\vec{P} = \vec{P}^{(0)}+\vec{P}^{(1)}+\vec{P}^{(2)}+\dots 
	\end{equation}
	où $\vec{P}^{(0)}$ est la polarisation statique, $\vec{P}^{(1)}$ est la polarisation 
	linéaire ($\propto \vec{E}$), \dots\\
	
	De façon générale, la polarisation en $(\vec{r},t)$ dépend d'autres position et temps
	que le doublet $(\vec{r},t)$. L'expression du $n^e$ ordre devient alors
	\begin{equation}
	\begin{array}{ll}
	\DS P_i^{(n)}(\vec{r},t) = \epsilon_0\iiint_{V_0}d\vec{r}_1\dots\iiint_{V_0}d\vec{r}_n
	\int_{-\infty}^\infty dt_1\dots&\DS \int_{-\infty}^\infty dt_n \chi_{ij_{1}\dots j_n}^{(n)}
	(\vec r, \vec r_1,\dots, \vec{r}_n, t, t_1,\dots,t_n)\\ &\DS\ E_{j1}(\vec{r}_1,t)\dots 
	E_{jn}(\vec{r}_n,t_n)
	\end{array}
	\end{equation}
	où $\chi^{(n)}$ est la susceptibilité d'ordre $n$, un tenseur de rang $n+1$. Nous faisons
	ici l'hypothèse que le matériau est homogène (propriétés identique en chaque point de 
	l'espace). En cas de discontinuité, on supposera que la théorie de continuité reste valide
	à l'interface. On fait également l'hypothèse que le matériau est stationnaire : la même 
	expérience faite aujourd'hui ou demain donnera les mêmes résultats. Il s'agit bien sur 
	d'approximations. Ceci s'exprime mathématiquement par un tenseur de susceptibilité qui ne
	dépend que des différences entre les coordonnées spatiales et temporelles.
	
	\begin{equation}
	\begin{array}{ll}
	\DS P_i^{(n)}(r,t) =\epsilon_0&\DS\iiint_{V_0}dr_1\dots \iiint_{V_0} dr_n \int_{-\infty}^
	\infty dt_1\int_{-\infty}^	\infty dt_n\\
	&\DS \chi_{ij_1\dots j_n}^{(n)}(r-r_1,\dots,r-r_n,t,\dots,t-t_n)E_{j_1}(r_1,t_1)\dots
	E_{j_n}(r_n,t_n)
	\end{array}
	\end{equation}